\documentclass[12pt]{article}

\usepackage[utf8]{inputenc}
\usepackage{graphicx}
\usepackage{float}
\usepackage{amsmath}
\usepackage{placeins}
\usepackage{gensymb}
\usepackage{caption}
\usepackage{subcaption}
\setcounter{section}{-1}


\usepackage[letterpaper,margin=0.75in]{geometry}
\graphicspath{images/}
\begin{document}
\begin{titlepage}
\begin{center}
% Upper part of the page
\vbox{}
\vbox{}
\vbox{}
\vbox{}
\vbox{}
\vbox{}
\vbox{}
\vbox{}
\vbox{}
\includegraphics[width=0.75\textwidth]{Images/ubc.png}\\[0.5cm]
\textrm{Martin Alejo}\\[0.5cm]
\catcode`#=12
\textrm{#75296665}\\[0.5cm]
\textrm{December 9, 2022}\\[0.5cm]
\textrm{Mini Project 4}\\[0.5cm]
\textrm{University of British Columbia}\\[0.5cm]
\textrm{Electrical and Computer Engineering}\\[0.5cm]
\textrm{ELEC 301}\\[0.5cm]
\textrm{Instructor: Nicolas Jaeger}\\[0.5cm]
\includegraphics[width=0.18\textwidth]{Images/Signature.png}\\[0.5cm]
\vbox{ }
\end{center}
\end{titlepage}
\pagebreak
\pagenumbering{roman}
\tableofcontents
\pagebreak
\listoffigures
\listoftables
\pagebreak
\pagenumbering{arabic}


\section{Introduction}
For this project, we will be using Multisim to simulate active filters and oscillators.
\section{Part A}
\subsubsection{Part 1}
For this part, we will be designing a 2nd order Butterworth low pass active filter using the UA741 operational amplifier. 
Here is the circuit that we will be using for this part:

\begin{figure}[H]
    \centering
    \includegraphics[height=0.2\textwidth]{Images/partAcircuit.png}\\
    \caption{Second Order Butterworth Filter}
    \label{fig:SecondOrderButterworthFilter}
\end{figure}

The calculations to find the resistance $R_1$,$R_2$ and capacitance C, we will be using the formulas from the class notes [1].  
The formulas can also be found from no. 1 in the Appendix. From the formulas, we can see that

\begin{center}
\boxed{R_1 = 6.306k \Omega, R_2 = 3.694k\Omega, C = 1.6nF, A_m = 1.59\frac{V}{V}}
\end{center}
Below is the phase and magnitude plot for our filter:

\begin{figure}[H]
    \centering
    \includegraphics[height=0.4\textwidth]{Images/magnitude_plot.png}\\
    \caption{Bode Magnitude Plot}
    \label{fig:magntitudeplot}
\end{figure}


\begin{figure}[H]
    \centering
    \includegraphics[height=0.4\textwidth]{Images/phase_plot.png}\\
    \caption{Bode Phase Plot}
    \label{fig:phaseplot}
\end{figure}


\subsubsection{Part 2}
For this part, we will be grounding the input, and measuring the output of the OpAmp.
To determine the value of $A_m$ when the circuit begins to oscillate, we need the transfer fucntion. The function is shown below,
where R = $10k\Omega$ and C is the value found previously:
\begin{flalign}
&H(s) = A_M\frac{\frac{1}{(RC)^2}}{s^2+s\frac{3-A_M}{RC}+\frac{1}{(RC)^2}}\nonumber
\end{flalign}
Changing the values of the resistances, we find that the oscillations occour when the resistor values are around$R_1=3k\Omega$ and $R_2 =
7k\Omega$.
The oscillation is shown below:
\begin{figure}[H]
    \centering
    \includegraphics[height=0.32\textwidth]{Images/partatransient.png}\\
    \caption{Oscillating Output}
    \label{fig:oscillatingoutput}
\end{figure}
\FloatBarrier
Using the cursors and measuring the differences between the crests of the plot, we find that the oscillating frequency is 
\boxed{f_o = 8.93kHz}.
Below is the root locus plot:

\begin{figure}
\centering
\begin{minipage}{.5\textwidth}
    \centering
    \includegraphics[height=0.7\textwidth]{Images/unstablerootlocus.png}\\
    \caption{Unstable Root Locus Plot}
    \label{fig:unstablerootlocus}
\end{minipage}%
\begin{minipage}{.5\textwidth}
    \centering
    \includegraphics[height=0.7\textwidth]{Images/stablerootlocus.png}\\
    \caption{Stable Root Locus Plot}
    \label{fig:rootlocus}
\end{minipage}
\end{figure}
\FloatBarrier
The root locus plot where $A_M>3$ and $A_M<3$ is Figure \ref{fig:unstablerootlocus} and Figure \ref{fig:rootlocus} respectively. 
We can see from the plot that when the oscillations occour, $A_M$ is greater than 3. This would then cause the system to be 
unstable as shown in Figure \ref{fig:unstablerootlocus}, since the poles are on the right side on the $jw$ axis. When the
poles are on the other side of the $jw$ axis, the system is stable, and doesn't cause the output to oscillate. This happens
when $A_M<3$. The reason why it has to be less than 3 for the system to be stable is because of the characteristic equation in the transfer function. 
If $A_M$ is equal or greater than 3, it causes one of the coefficients in the characteristic equation to be negative, causing instability
in the system and thus, the oscillations occour.

\section{Part B}
Below is our phase shift oscillator circuit:
\begin{figure}[H]
    \centering
    \includegraphics[height=0.2\textwidth]{Images/phaseoscillatorcircuit.png}\\
    \caption{Phase Shift Oscillator Circuit}
    \label{fig:oscillatorcircuit}
\end{figure}
\FloatBarrier

To find the proper value of the 29R resistor, we will simulate the output response for a very large amount of time (in this case, 1000 seconds
was used). Simulating with just $29k\Omega$ we find that the output amplitude eventually decays to zero. Increasing the resistor
to $29.1k\Omega$ makes the output amplitude to be sustained indefinitely. 

Plotting the output of the circuit with the values given in the project document [2] and the increased 29R resistor , we find that 
the output oscillates as shown: 

\begin{figure}[H]
    \centering
    \includegraphics[height=0.3\textwidth]{Images/partbtransient1x.png}\\
    \caption{Circuit Output With Unchanged Circuit Elements}
    \label{fig:oscillatorcircuit1x}
\end{figure}

\begin{figure}[H]
    \centering
    \includegraphics[height=0.3\textwidth]{Images/partbtransient0.5x.png}\\
    \caption{Circuit Output With Half Circuit Elements}
    \label{fig:oscillatorcircuit1x}
\end{figure}

\begin{figure}[H]
    \centering
    \includegraphics[height=0.3\textwidth]{Images/partbtransient2x.png}\\
    \caption{Circuit Output With Double Circuit Elements}
    \label{fig:oscillatorcircuit1x}
\end{figure}
From class notes[1], we can calculate the output frequency with:
\begin{flalign}
    &f=\frac{1}{2\pi RC \sqrt{6}} \nonumber
\end{flalign} 

Comparing the calculated and measured frequencies for each modified circuit:

\begin{table}[]
    \centering
    \resizebox{0.5\textwidth}{!}{%
    \begin{tabular}{l|l|l|l}
    
    R and C Multiplier & 0.5x & 1x & 2x \\ \cline{1-4}
    Calculated $f_o(Hz)$ & 259.899 & 64.975 & 16.244 \\ \cline{1-4}
    Measured $f_o(Hz)$ & 259.067 & 64.872 & 16.221 \\ \cline{1-4}
    \% Error & 0.32 & 0.15 & 0.14
    \end{tabular}%
    }
    \caption{Calculated and Measured Frequencies}
    \label{CalculatedMeasuredFrequencies}
\end{table}
\FloatBarrier
We can observe that the calculated and measured values are very close to each other.

\section{Part C}
Below is our Feedback Circuit:
\begin{figure}[H]
    \centering
    \includegraphics[height=0.25\textwidth]{Images/partCcircuit.png}\\
    \caption{Feedback Circuit}
    \label{fig:feedbackcircuit}
\end{figure}

Since we want to find the value of $R_{B2}$ that creates the largest open-loop gain at 1kHz, we will be setting our
input voltage at 1kHz at 1mV. To find the value, we will be doing a parameter sweep with a transient response to 
find which resistance value results in the largest gain. We find that after doing the parameter sweep,
the value $R_{B2}$ to be around \boxed{20k\Omega}. Resistance values above $20k\Omega$
, we find that the output amplitude decreases. The paramater sweep is shown below:

\begin{figure}[h!]
    \centering
    \includegraphics[height=0.4\textwidth]{Images/parametersweep.png}\\
    \caption{Parameter Sweep with $R_{B2}$ and Transient Response}
    \label{fig:parametersweep}
\end{figure}


\subsection{Part 1}
To measure the DC Operating points, we will be using the DC Circuit below:
\begin{figure}[h!]
    \centering
    \includegraphics[height=0.25\textwidth]{Images/partCDC.png}\\
    \caption{Feedback DC Circuit}
    \label{fig:feedbacdccircuit}
\end{figure}
\FloatBarrier
For this project, we will be assuming that $V_T$= 0.025V. We can use these formulas to 
calculate our transistor parameters:
\begin{center}
    $r_\pi= \frac{V_T}{I_B}$ ,$\beta = \frac{I_C}{I_B}$ and $g_m = \frac{\beta}{r_\pi}$
\end{center}
Here are the DC Operating points, with our measured $r_\pi,g_m$, and $h_{FE}$:
\FloatBarrier
\begin{table}[h!]
    \centering
    \begin{tabular}{l|lllllllll}
     & $I_C$ & $I_B$ & $I_E$ & $V_C$ & $V_B$ & $V_E$ & $g_m$ & $r_\pi$ & $h_{FE}$ \\ \cline{1-10}
    Q1 & 1.29mA & 10.8uA & 1.31mA & 1.90V & 0.654V & 0V & 0.05 & 2.315k$\Omega$ & 119  \\ \cline{1-10}
    Q2 & 2.19mA & 15.4uA & 2.21mA & 15V & 1.90V & 1.23V & 0.09 & 1.623k$\Omega$ & 142
    \end{tabular}%
    \caption{DC Operating Points}
    \label{DC Operating Points}
\end{table}

\subsection{Part 2}
\subsubsection{Finding Measured Open Loop Frequency Responce and I/O Impedance}
Here is the open loop frequency response plot:
\begin{figure}[h!]
    \centering
    \includegraphics[height=0.35\textwidth]{Images/partCbode.png}\\
    \caption{Open Loop Frequency Response}
    \label{fig:olfreqresponse}
\end{figure}
\FloatBarrier
Measuring with the cursors in our simulation software, we find that the 
3dB points are: 
\begin{center}
    \boxed{w_{L3dB} = 2.9219 *2\pi [\frac{rad}{s}], w_{H3dB} = 91.3083*2\pi*10^3[\frac{rad}{s}] } 
\end{center}
We can also find our mid-band gain \boxed{A_m = 127.557\frac{V}{V}}.
In order to find the input and output resistance at 1kHz, we will be measuring the voltage 
and current at the input, and we will be adding a test source at the output and grounding the
input for the output impedance measurements. After measuring, we find: 
\begin{center}
$R_{in} = \frac{240\mu V}{93.4nA}$ = \boxed{2.569k\Omega}, 
$R_{out} = \frac{707\mu V}{11.4 \mu A} = $ \boxed{62.017\Omega}
\end{center}

\subsubsection{Predicted Closed Loop Frequency Response}
For this section, we will need to calculate our circuit parameters and input and 
output resistance at 1kHz with $R_f=100k\Omega$. To do this, we need to know our circuit's 
topology. Since the output is sampled by $R_f$ and mixes the current from the output into the
input, we can determine that our feedback circuit uses shunt-shunt topology. Due to this,
we will be using y-parameters to perform our calculations. Below is the y-parameter matrix
we will be using:
\begin{center}
$\begin{bmatrix}
        I_1 \\
        I_2
\end{bmatrix}$ =
$\begin{bmatrix}
    V_{1} \\
    V_{2}  
\end{bmatrix}$ 
$\begin{bmatrix}
    y_{11} && y_{12} \\
    y_{21} && y_{22} 
\end{bmatrix}$
\end{center}

In this case, we only need our $\beta$ value from our y-parameters, therefore we can just
solve for that value
\subsubsection{Gain Calculations}
Since our circuit is in shunt-shunt topology, our open-loop will be  $\frac{V_{out}}
{I_{in}}$. Therefore, we need to convert our input to the amplifier from a voltage source 
to a current source. To do this, we will be transforming our source from a voltage source 
in series with our source resistance to a current source in parallel to our source resistance. 
Doing this, our resulting input current is \boxed{I_{in} = \frac{V_s}{5k\Omega}}. So we find
our open loop gain to be:
\begin{flalign}
    &A^{'}= \frac{V_{out}}{I_{in}} = \frac{V_{out}}{\frac{V_s}{5k\Omega}}  = \frac{V_{out}}
    {V_{s}}(5k\Omega) = 127.557\frac{V}{V}*(5k\Omega) = \boxed{637.786k\frac{V}{A}} \nonumber
\end{flalign}

Now, we are able to find our gain with feedback. The formula used
can be found in the class notes[1]. It is found as:
\begin{flalign}
&A_f = \frac{A^{'}}{1+\beta A^{'}} = 86.446k\frac{V}{A} \nonumber
\end{flalign}
Since the units we need is $\frac{V}{V}$ we convert our input to a voltage source,
and this can be done by dividing $A_f$ by $5k\Omega$. After doing this, we find
\boxed{A_f = 17.289 \frac{V}{V}}. Which is our closed loop gain.

\subsubsection{3dB Calculations}
To calculate the $w_{3dBf}$ (3dB frequencies with feedback), we can use the 
following formulas from the class notes[1]. The frequencies are found to be:
\begin{flalign}
&w_{Hf3dB} = (w_{H3dB})(1+A^{'}\beta) = (91.308k*2\pi)(1+\frac{1}{100k\Omega}637.786k\frac{V}{A}) = \boxed{ 4233.189k \frac{rad}{s}}\nonumber\\
&w_{Lf3dB} = \frac{w_{L3dB}}{1+A^{'}\beta} = \frac{2.9219 *2\pi}{1+\frac{1}{100k\Omega}(637.786k\frac{V}{A})} =\boxed{2.488 \frac{rad}{s}} \nonumber
\end{flalign}

\subsubsection{Input and Output Resistance Calculations}
To calculate our input and output resitance values with feedback, we can 
use our open loop resistance values and formulas[1] to calculate the values
we eventually find:
\begin{flalign}
&R_{if} =  \frac{R_{in}}{1+A^{'}\beta} = \frac{2.569k\Omega}{1+\frac{1}{100k\Omega}(637.786k\frac{V}{A})} =\boxed {348.204\Omega}\nonumber \\
&R_{of} = \frac{R_{out}}{1+A^{'}\beta} = \frac{62.017\Omega}{1+\frac{1}{100k\Omega}(637.786k\frac{V}{A})} =  
\boxed{8.406\Omega}\nonumber
\end{flalign}

Here is our closed loop frequency response curve:

\begin{figure}[h!]
    \centering
    \includegraphics[height=0.4\textwidth]{Images/partcfeedbackbode.png}\\
    \caption{Closed Loop Frequency Response}
    \label{fig:clfreqresponse}
\end{figure}
From this, we can measure the following values:

\begin{table}[h!]
    \centering
    \begin{tabular}{l|lllll}
        
     & $w_{L3dB}$ & $w_{H3dB}$ & $R_{in}$ & $R_{out}$ & $A_M$ \\ \cline{1-6}
    Measured & 3.221[$\frac{rad}{s}$] & 4257.361k[$\frac{rad}{s}$] & $240.7\Omega$ & $8.456\Omega$ & 17.248$\frac{V}{V}$ \\ 
    \end{tabular}   
    \caption{Measured Feedback Values}
    \label{Measuredfeedbackvalues}
\end{table}

Comparing the measured values and the calculated values, we can see that the calculated 
values are accurate to the measured values. 

\subsection{Part 3}
From the bottom most trace, the values of $R_f$ are increasing for each increasing curve. 
Here is all of the different bode plots of various $R_f$ values:
\begin{figure}[h!]
    \centering
    \includegraphics[height=0.4\textwidth]{Images/partcbode1k.png}\\
    \caption{Closed Loop Frequency Response with Different $R_f$ Values}
    \label{fig:clfreqresponse}
\end{figure}

Comparing the values as shown in the previous part, all of the estimates are close to the measured
values.




\subsubsection{Part 4}





\newpage
\section{Appendix}
\begin{flalign}
    &k = 3 - \sqrt{2}\\
\end{flalign}
\section{References}
1. ELEC 301 Class notes 
\newline
2. Mini Project 4 Document
\newline
3. Standard Resistor and Capacitor Values (Canvas)
\newline
4. Circuit Maker SPICE Model


\end{document}